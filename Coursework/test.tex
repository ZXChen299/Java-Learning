\documentclass[a4paper,11pt]{article} %%%%%%%%%%%% start of LaTeX file
\usepackage{mathpazo}
\usepackage{tikz}
\usetikzlibrary{shapes}
\oddsidemargin -0.54cm
\textwidth 17.0cm
\textheight 24cm
\topmargin -1.3cm
\parindent 0pt
\parskip 1ex
\pagestyle{empty}
\begin{document} %%%%%%%%%%%% end of LaTeX preamble, start of text
	\medskip\hrule\medskip
	\begin{tikzpicture}[scale=2.125]
	\draw [help lines, color=green] (0,0) grid (8,5) ;
	\draw [thick] (0,0) node[rounded rectangle] (1) {4};
	\draw [thick] (1,1) node[draw, rounded rectangle] (2) {2};
	\draw [thick] (2,0) node[draw, rounded rectangle] (3) {3};
	\draw [thick] (0,0) node[draw, rounded rectangle] (4) {4};
	% coordinate 150 in node (150,150) too large
	\draw [thick] (4,0) node[draw, rounded rectangle] (6) {6};
	\draw [thick] (5,5) node[draw, rounded rectangle] (7) {7};
	% coordinate 160 in node (160,1) too large
	\draw [thick] (8,5) node[rounded rectangle] (9) {10};
	\draw [thick] (8,5) node[draw, rounded rectangle] (10) {10};
	\draw [->, thick] (1) to (2);
	\draw [->, thick] (2) to (3);
	\draw [->, thick] (3) to (4);
	\draw [->, thick] (6) to (7);
	\draw [->, thick] (9) to (10);
	\end{tikzpicture}
	\medskip\hrule\medskip
\end{document}
